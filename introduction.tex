\section{Introduction}

\subsection{Book I}

Tell me, O Muse, of that ingenious hero who travelled far and wide after he
had sacked the famous town of Troy. Many cities did he visit, and many were
the nations with whose manners and customs he was acquainted; moreover he
suffered much by sea while trying to save his own life and bring his men
safely home; but do what he might he could not save his men, for they perished
through their own sheer folly in eating the cattle of the Sun-god Hyperion; so
the god prevented them from ever reaching home. Tell me, too, about all these
things, oh daughter of Jove, from whatsoever source you may know them.

So now all who escaped death in battle or by shipwreck had got safely home
except Ulysses, and he, though he was longing to return to his wife and country,
 was detained by the goddess Calypso, who had got him into a large cave and
 wanted to marry him. But as years went by, there came a time when the gods
 settled that he should go back to Ithaca; even then, however, when he was
 among his own people, his troubles were not yet over; nevertheless all the
 gods had now begun to pity him except Neptune, who still persecuted him
 without ceasing and would not let him get home.

Now Neptune had gone off to the Ethiopians, who are at the world's end, and
lie in two halves, the one looking West and the other East. He had gone
there to accept a hecatomb of sheep and oxen, and was enjoying himself at
his festival; but the other gods met in the house of Olympian Jove, and the
sire of gods and men spoke first. At that moment he was thinking of Aegisthus,
who had been killed by Agamemnon's son Orestes; so he said to the other gods:

"See now, how men lay blame upon us gods for what is after all nothing but
their own folly. Look at Aegisthus; he must needs make love to Agamemnon's
wife unrighteously and then kill Agamemnon, though he knew it would be the
death of him; for I sent Mercury to warn him not to do either of these things,
inasmuch as Orestes would be sure to take his revenge when he grew up and
wanted to return home. Mercury told him this in all good will but he would not
listen, and now he has paid for everything in full."

Then Minerva said, "Father, son of Saturn, King of kings, it served Aegisthus
right, and so it would any one else who does as he did; but Aegisthus is
neither here nor there; it is for Ulysses that my heart bleeds, when I think
of his sufferings in that lonely sea-girt island, far away, poor man, from
all his friends. It is an island covered with forest, in the very middle of
the sea, and a goddess lives there, daughter of the magician Atlas, who looks
after the bottom of the ocean, and carries the great columns that keep heaven
and earth asunder. This daughter of Atlas has got hold of poor unhappy Ulysses,
and keeps trying by every kind of blandishment to make him forget his home, so
that he is tired of life, and thinks of nothing but how he may once more see
the smoke of his own chimneys. You, sir, take no heed of this, and yet when
Ulysses was before Troy did he not propitiate you with many a burnt sacrifice?
Why then should you keep on being so angry with him?"

\subsection{Book II}

And Jove said, "My child, what are you talking about? How can I forget Ulysses
 than whom there is no more capable man on earth, nor more liberal in his
 offerings to the immortal gods that live in heaven? Bear in mind, however,
 that Neptune is still furious with Ulysses for having blinded an eye of
 Polyphemus king of the Cyclopes. Polyphemus is son to Neptune by the nymph
 Thoosa, daughter to the sea-king Phorcys; therefore though he will not kill
 Ulysses outright, he torments him by preventing him from getting home. Still,
 let us lay our heads together and see how we can help him to return; Neptune
 will then be pacified, for if we are all of a mind he can hardly stand out
 against us."

And Minerva said, "Father, son of Saturn, King of kings, if, then, the gods
now mean that Ulysses should get home, we should first send Mercury to the
Ogygian island to tell Calypso that we have made up our minds and that he is
to return. In the meantime I will go to Ithaca, to put heart into Ulysses' son
Telemachus; I will embolden him to call the Achaeans in assembly, and speak
out to the suitors of his mother Penelope, who persist in eating up any number
of his sheep and oxen; I will also conduct him to Sparta and to Pylos, to see
if he can hear anything about the return of his dear father—for this will
make people speak well of him."

So saying she bound on her glittering golden sandals, imperishable, with which
she can fly like the wind over land or sea; she grasped the redoubtable
bronze-shod spear, so stout and sturdy and strong, wherewith she quells the
ranks of heroes who have displeased her, and down she darted from the topmost
summits of Olympus, whereon forthwith she was in Ithaca, at the gateway of
Ulysses' house, disguised as a visitor, Mentes, chief of the Taphians, and
she held a bronze spear in her hand. There she found the lordly suitors seated
on hides of the oxen which they had killed and eaten, and playing draughts in
front of the house. Men-servants and pages were bustling about to wait upon
them, some mixing wine with water in the mixing-bowls, some cleaning down the
tables with wet sponges and laying them out again, and some cutting up great
quantities of meat.

Telemachus saw her long before any one else did. He was sitting moodily among
the suitors thinking about his brave father, and how he would send them flying
out of the house, if he were to come to his own again and be honoured as in
days gone by. Thus brooding as he sat among them, he caught sight of Minerva
and went straight to the gate, for he was vexed that a stranger should be kept
waiting for admittance. He took her right hand in his own, and bade her give
him her spear. "Welcome," said he, "to our house, and when you have partaken
of food you shall tell us what you have come for."

He led the way as he spoke, and Minerva followed him. When they were within he
took her spear and set it in the spear-stand against a strong bearing-post
along with the many other spears of his unhappy father, and he conducted her
to a richly decorated seat under which he threw a cloth of damask. There was a
footstool also for her feet, and he set another seat near her for himself,
away from the suitors, that she might not be annoyed while eating by their
noise and insolence, and that he might ask her more freely about his father.

A maid servant then brought them water in a beautiful golden ewer and poured
it into a silver basin for them to wash their hands, and she drew a clean
table beside them. An upper servant brought them bread, and offered them many
good things of what there was in the house, the carver fetched them plates of
 all manner of meats and set cups of gold by their side, and a manservant
 brought them wine and poured it out for them.
